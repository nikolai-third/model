\documentclass[a4paper,12pt]{article}

\usepackage[english,russian]{babel}
\usepackage{cmap}
\usepackage[T2A]{fontenc}
\usepackage[utf8]{inputenc}
\usepackage{amsmath}
\usepackage{multirow }
\usepackage{pgfplots}
\usepackage{wrapfig}
\usepackage{graphicx}
\graphicspath{{./images/}}
\usepackage{caption}
\usepackage{subcaption}
\usepackage{longtable}
\usepackage{xcolor}
\usepackage{ gensymb }
\usepackage{ dsfont }
\usepackage[unicode, pdftex]{hyperref}
\hypersetup{colorlinks,
	pdftitle={The title of your document},
	pdfauthor={Your name},
	allcolors=[RGB]{0 0 245}}


\textwidth=7.3in % ширина текста
\textheight=10in % высота текста
\oddsidemargin= -0.5in % левый отступ(базовый 1дюйм + значение)
\topmargin= -0.5in % отступ сверху до колонтитула(базовый 1дюйм + значение)


\author{Победин Н. К. Б02-212}
\title{Небольшой лит. обзор статей по теме "Связь стабильности $\alpha$-спирали с ее аминокислотным составом"}
\date{\today}

\begin{document}
	\maketitle
	Изначально процесс строился так: люди брали один небольшой белок, мутировали в нем одну конкретную позицию разными по свойствам(заряженные/полурные/гиброфобные) аминокислотами и проверяли стабильнее становится белок или нет. И в \cite{litlink1} приходят к выводу, что на стабильность $\alpha$-спирали влияют не только конформации аминокислот, но и расположение $C$- и $N$-концов , а также другие взаимодействия как внутри спирали, так и с остальной частью белка и изменения воздействия растворителя при сворачивании. 
	
	В \cite{litlink2} структурная параметризация энергии сворачивания была использована для прогнозирования влияния мутаций отдельных аминокислот на открытые участки $\alpha$-спиралей. Результаты были использованы для получения термодинамической шкалы сродства $\alpha$-спиралей к аминокислотам, основанной на структуре. Структурный термодинамический анализ был проведен для четырех различных систем, для которых доступны структурные и экспериментальные термодинамические данные. 
	Эти исследования позволили оптимизировать набор доступных для растворителя участков поверхности для всех аминокислот в несвернутом состоянии. Показано, что единый набор структурных/термодинамических параметров хорошо учитывает все экспериментальные данные о различных вариантах чувствительности. Для лизоцима среднее значение абсолютной разницы между прогнозируемыми и экспериментальными значениями $\Delta G$ составляет 0,09 ккал/моль, для барназы - 0,14 ккал/моль, для синтетической спиральной катушке - 0,11 ккал/моль и для синтетического пептида - 0,08 ккал/моль. Кроме того, этот подход хорошо предсказывает общую стабильность белков и объясняет различия в сродстве аминокислот к $\alpha$-спирали. Наблюдаемое соответствие между прогнозируемыми и экспериментальными значениями $\Delta G$ для всех аминокислот подтверждает рациональность использования этой структурной параметризации в расчетах свободной энергии для сворачивания или связывания.
	
	В \cite{litlink3} Обычно считается, что спиральные белки стабилизируются за счет сочетания гидрофобных и упаковочных взаимодействий, а также водородных связей и электростатических взаимодействий.В статье показывается, что взаимодействия поляризованных боковых цепей на внешней части белка могут играть важную роль в формировании спирали и ее стабильности. Рассматривается исследования модельных спиральных пептидов, которые выявляют влияние слабых взаимодействий между боковыми цепями на стабильность спирали, уделяя особое внимание некоторым неклассическим взаимодействиям между боковыми цепями: сложным солевым мостикам, катионным и C—H$\cdot \cdot \cdot$O-H-связям. Также обсуждается вопрос о структуре релаксированных состояний спиральных пептидов в свете некоторых экспериментов, показывающих, что они содержат значительные количества конформации полипролина II.
	




\begin{thebibliography}{}
	\bibitem{litlink1}  Amnon Horovitzj-, Jacqueline M. Matthews and Alan R. Fersht \textit{a-Helix Stability in Proteins II. Factors that Influence Stability at an Internal Position}. 
	
	\bibitem{litlink2}  Irene Luque, Obdulio L. Mayorga, and Ernesto Freire
	\textit{Structure-Based Thermodynamic Scale of R-Helix Propensities in Amino Acids.}
	
	\bibitem{litlink3}  Zhengshuang Shi C. Anders Olson Anthony J. Bell, Jr. Neville R. Kallenbach
	\textit{Stabilization of $\alpha$-Helix Structure by Polar Side- Chain Interactions: Complex Salt Bridges, Cation–$\pi$ Interactions, and $C - H \cdot \cdot \cdot O H$-Bonds} 
	

	

\end{thebibliography}
	
\end{document}
